\chapter{Localization of Equivalent Current Dipoles \label{Chap:eeg:VBECD}}

This chapter describes source reconstruction based on ``Variational Bayes Equivalent Current Dipoles'' (VB-ECDs).
For more details about the implementation, please refer to the help and comments in the routines themselves, as well as the original paper by \cite{stefan_vb_ecd}.

\section{Introduction}
3D imaging (or distributed) reconstruction methods consider all possible source location simultaneously, allowing for large and widely spread clusters of activity. This is to be contrasted with ``Equivalent Current Dipole�� (ECD) approaches which rely on two different hypotheses:
\begin{itemize}
\item only a few (say less than~5) sources are active simultaneously, and
\item those sources are very focal.
\end{itemize}
This leads to the ECD model where the observed scalp potential will be explained by a handful of discrete current sources, i.e. dipoles, located inside the brain volume.

In contrast to the 3D imaging reconstruction, the number of ECDs considered in the model, i.e. the number of ``active locations��, should be defined a priori. This is a crucial step, as the number of sources considered defines the ECD model. This choice should be based on empirical knowledge of the brain activity observed or any other source of information (for example by looking at the scalp potential distribution).
In general, each dipole is described by 6 parameters: 3 for its location, 2 for its orientation and 1 for its amplitude. Once the number of ECDs is fixed, a non-linear optimisation algorithm is used to adjust the dipoles parameters (6 times the number of dipoles) to the observed potential.

Classical ECD approaches use a simple best fitting optimisation using ``least square error'' criteria. This leads to relatively simple algorithms but presents a few drawbacks:
\begin{itemize}
\item constraints on the dipoles are difficult to include in the framework;
\item the noise cannot be properly taken into account, as its variance should be estimated alongside the dipole parameters;
\item it is difficult to define confidence intervals on the estimated parameters, which could lead to over-confident interpretation of the results;
\item models with different numbers of dipoles cannot be compared except through their goodness-of-fit, which can be misleading.
\end{itemize}
As adding dipoles to a model will necessarily improve the overall goodness of fit, one could erroneously be tempted to use as many ECDs as possible and to perfectly fit the observed signal.
\\
Through using Bayesian techniques, however, it is possible to circumvent all of the above limitations of classical approaches.

Briefly, a probabilistic generative model is built providing a likelihood model for the data\footnote{This includes an independent and identically distributed (IID) Normal distribution for the errors, but other distributions could be specified.}. The model is completed by a set of priors on the various parameters, leading to a Bayesian model, allowing the inclusion of user-specified prior constraints. 

A ``variational Bayes�� (VB) scheme is then employed to estimate the posterior distribution of the parameters through an iterative procedure. The confidence interval of the estimated parameters is therefore directly available through the estimated posterior variance of the parameters.
Critically, in a Bayesian context, different models can be compared using their evidence or marginal likelihood. This model comparison is superior to classical goodness-of-fit measures, because it takes into account the complexity of the models (e.g., the number of dipoles) and, implicitly, uncertainty about the model parameters. VB-ECD can therefore provide an objective and accurate answer to the question: Would this data set be better modelled by 2 or 3 ECDs?

\section{Procedure in SPM12}
This section aims at describing how to use the VB-ECD approach in SPM12.

\subsection{Head and forward model}
The engine calculating the projection of the dipolar sources on the scalp electrode comes from Fieldtrip and is the same for the 3D imaging or DCM. The head model should thus be prepared the same way, as described in the chapter \ref{Chap:eeg:imaging}. For the same data set, differences  between the VB-ECD and imaging reconstructions would therefore be due to the reconstruction approach only.

\subsection{VB-ECD reconstruction}
To get started, after loading and preparing the head model, press the 'Invert' button\footnote{The GUI for VB-ECD can also be launched directly from \matlab\ command line with the instruction: \texttt{D = spm\_eeg\_inv\_vbecd\_gui}.}. The first choice you will see is between 'Imaging', 'VB-ECD' and 'DCM'. The 'Imaging' reconstruction corresponds to the imaging solution, as described in chapter \ref{Chap:eeg:imaging}, and 'DCM' is described in chapter \ref{Chap:eeg:DCM}.
Then you are invited to fill in information about the ECD model and click on buttons in the following order:
\begin{enumerate}
\item indicate the time bin or time window for the reconstruction, within the epoch length. Note that the data will be averaged over the selected time window! VB-ECD will thus always be calculated for a single time bin.
\item enter the trial type(s) to be reconstructed. Each trial type will be reconstructed separately.
\item \label{add_dip} add a single (i.e. individual) dipole or a pair of symmetric dipoles to the model. Each ``element�� (single or pair) is added individually to the model.
\item use ``Informative�� or `Non-informative�� location priors. ``Non-informative�� means flat priors over the brain volume. With ``Informative��, you can enter the a priori location of the source\footnote{For a pair of dipoles, only the right dipole coordinates are required.}.
\item use ``Informative�� or `Non-informative�� moment priors. ``Non-informative�� means flat priors over all possible directions and amplitude. With ``Informative��, you can enter the a priori moment of the source\footnote{For a pair of dipoles, only the right dipole moment is required.}.
\item go back to step \ref{add_dip} and add some more dipole(s) to the model, or stop adding dipoles.
\item specify the number of iterations. These are repetitions of the fitting procedure with different initial conditions. Since there are multiple local maxima in the objective function, multiple iterations are necessary to get good results especially when non-informative location priors are chosen.
\end{enumerate}
The routine then proceeds with the VB optimization scheme to estimate the model parameters. There is graphical display of the intermediate results. When the best solution is selected the model evidence will be shown at the top of the SPM Graphics window. This number can be used to compare solutions with different priors.
\\
Results are finally saved into the data structure \texttt{D} in the field \texttt{.inv\{D.val\}.inverse} and displayed in the graphic window.

\subsection{Result display}

The latest VB-ECD results can be displayed again through the function \texttt{D = spm\_eeg\_inv\_vbecd\_disp}. If a specific reconstruction should be displayed, then use: \texttt{spm\_eeg\_inv\_vbecd\_disp('Init',D, ind)}. In the GUI you can use the \texttt{'dip'} button (located under the 'Invert' button) to display the dipole locations.
\\
In the upper part, the 3 main figures display the 3 orthogonal views of the brain with the dipole location and orientation superimposed. The location confidence interval is described by the dotted ellipse around the dipole location on the 3 views. It is not possible to click through the image, as the display is automatically centred on the dipole displayed. It is possible though to zoom into the image, using the right-click context menu.

The lower left table displays the current dipole location, orientation (Cartesian or polar coordinates) and amplitude in various formats.

The lower right table allows for the selection of trial types and dipoles. Display of multiple trial types and multiple dipoles is also possible. The display will center itself on the average location of the dipoles.





